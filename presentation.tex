\documentclass{article}
\usepackage[T1]{fontenc}
\usepackage{amssymb, amsmath, graphicx, subfigure}

\setlength{\oddsidemargin}{.25in}
\setlength{\evensidemargin}{.25in}
\setlength{\textwidth}{6in}
\setlength{\topmargin}{-0.4in}
\setlength{\textheight}{8.5in}

\newcommand{\heading}[6]{
  \renewcommand{\thepage}{#1-\arabic{page}}
  \noindent
  \begin{center}
  \framebox{
    \vbox{
      \hbox to 5.78in { \textbf{#2} \hfill #3 }
      \vspace{4mm}
      \hbox to 5.78in { {\Large \hfill #6  \hfill} }
      \vspace{2mm}
      \hbox to 5.78in { \textit{Instructor: #4 \hfill #5} }
    }
  }
  \end{center}
  \vspace*{4mm}
}

\newtheorem{theorem}{Theorem}
\newtheorem{definition}[theorem]{Definition}
\newtheorem{remark}[theorem]{Remark}
\newtheorem{lemma}[theorem]{Lemma}
\newtheorem{corollary}[theorem]{Corollary}
\newtheorem{proposition}[theorem]{Proposition}
\newtheorem{claim}[theorem]{Claim}
\newtheorem{observation}[theorem]{Observation}
\newtheorem{fact}[theorem]{Fact}
\newtheorem{assumption}[theorem]{Assumption}

\newenvironment{proof}{\noindent{\bf Proof:} \hspace*{1mm}}{
	\hspace*{\fill} $\Box$ }
\newenvironment{proof_of}[1]{\noindent {\bf Proof of #1:}
	\hspace*{1mm}}{\hspace*{\fill} $\Box$ }
\newenvironment{proof_claim}{\begin{quotation} \noindent}{
	\hspace*{\fill} $\diamond$ \end{quotation}}

\newcommand{\problemset}[3]{\heading{#1}{CS276: Cryptography}{#2}{Alessandro Chiesa}{#3}{Problem Set #1}}




\newcommand\abs[1]{\left|#1\right|}
\newcommand\given[1][]{\:#1\vert\:}
\newcommand{\legendre}[2]{\genfrac{(}{)}{}{}{#1}{#2}}
%%%%%%%%%%%%%%%%%%%%%%%%%%%%%%%%%%%%%%%%%%%%%%%%%%%%%%%%%%%%%%%%%%%%%%%%%%%%%%%


%%%%%%%%%%%%%%%%%%%%%%%%%%%%%%%%%%%%%%%%%%%%%%%%%%%%%%%%%%%%%%%%%%%%%%%%%%%%%%%
\begin{document}
$E: y^2 = x^3 + Ax + B$
\bigskip

$-16(4A^3+27B^2) \ne 0$

\bigskip

$P \mapsto n P$
\bigskip

$1728\frac{A^3}{4A^3+27B^2}$

\bigskip


\noindent A pairing is a nondegenerate map $e: G_1 \times G_2 \to G_3$, where $G_i$ are cyclic and $\abs{G_i} = p$ satisfying:
\begin{enumerate}
\item $e(aP, bQ) = e(P,Q)^{ab}$
\item $e(P,Q) \ne 1$ for some $P,Q$
\item $e$ is efficiently computable
\end{enumerate}

\bigskip

The Weil Pairing: $e: E(F_p)[r] \times E(F_p)[r] \to \mu_r$, where $E(F_p)[r]$ is the group of $r$-torsion points and $\mu_r \subset \overline{F_p}$ are the $r$th roots of unity.
\bigskip

The Tate Pairing: $\tau: E(F_p)[r] \times E(F_p)[r]/ rE(F_p) \to \mu_r$

\bigskip

The Weil/Tate Pairing is efficiently computable when $\mu_r \subset F_{p^k}$, where $k$ is small. This holds iff $\gcd(r,p^k-1) = r \iff p^k-1 \equiv 0 \mod r \iff p$ is a primitive $k$th root of unity mod $r$. A curve with efficiently computable pairings is {\it pairing-friendly}.

\bigskip

If $r \approx p$, then $\Pr[$pairing-friendly curve$] = O(\frac{\log^3 M}{M})$
\end{document}
%%%%%%%%%%%%%%%%%%%%%%%%%%%%%%%%%%%%%%%%%%%%%%%%%%%%%%%%%%%%%%%%%%%%%%%%%%%%%%%